\documentclass[12pt]{article}

\usepackage{amsmath}
\usepackage{amsfonts}
\usepackage{amssymb}
\usepackage{amsthm}

% Define norm and abs
\newcommand{\abs}[1]{\lvert#1\rvert}
\newcommand{\norm}[1]{\lVert#1\rVert}

%\renewcommand{\emph}{\emph{\bf }}

% Define Theorem, Lemma, Corollary
\swapnumbers
\theoremstyle{definition}
\newtheorem{thm}{Theorem}[section]
\newtheorem{cor}[thm]{Corollary}
\newtheorem{lem}[thm]{Lemma}
\newtheorem{defn}[thm]{Definition}
\newtheorem{prop}[thm]{Proposition}
\newtheorem*{rem}{Remark}
\newtheorem{examp}[thm]{Example}

% Define Z, R, Q, C, etc. From Dr. Avramidi
\def\II{\mathbb{I}}
\def\RR{\mathbb{R}}
\def\ZZ{\mathbb{Z}}
\def\CC{\mathbb{C}}
\def\QQ{\mathbb{Q}}
\def\NN{\mathbb{N}}
\def\FF{\mathbb{F}}

\newcommand{\ang}[1]{\ensuremath{\left\langle #1 \right\rangle}}

\DeclareMathOperator{\spann}{span}
\newcommand\spanset[1]{\ensuremath\spann(#1)}

\title{Needham - Visual Complex Analysis Notes \date{\today}}
\author{Sean D. Turner, Hugo Rivera}

\begin{document}
\maketitle

\section{Mathematical Background}

\subsection{Sets}

A \emph{set} is a collection of objects. The empty set contains no elements. The universal set is the \emph{universe} of your problem. A subset $A \subseteq B$ means that every element of $A$ is an element of $B$. A \emph{proper subset} $A \subset B$ means that there is at least one element in $B$ that is not in $A$.

\subsection{Intersection and Union}
\subsection{Differences}
\subsection{Complements}
\subsection{De Morgan's Laws}
$$\overline{\bigcup A_i} = \bigcap \overline{A_i}$$ and $$\overline{\bigcap A_i} = \bigcup \overline{A_i}\,.$$

\subsection{Maps}

We have $$\varphi : A \to B$$ where $A$ is the codomain and $B$ is the domain. 

\subsection{Image and Inverse Image}

$$\varphi(X) = \{y \in B \, | \, y = \varphi(x) \text{ for some } x \in X\}$$ and

$$\varphi^{-1}(Y) = \{x \in A \, | \, \varphi(x) \in Y\}\, . (X \subset A, Y \subset B)\, .$$

\begin{defn}\label{def0-0}
The \emph{image} (or range) of a function; $Im{\varphi} = \varphi(A)$.
\end{defn}

\subsection{Injective, Surjective, and Bijective Maps}

\subsection{Permutation}

A \emph{permutation} is a bijection of a finite set. 

\subsection{Cardinality}

$\abs{A} = \abs{B}$ if there exists a bijection $\varphi : A \to B$.

\subsection{Finite Sets}

A set $X$ is \emph{finite} if $\abs{X} \abs{\ZZ_n} = n$. Then, $\abs{X} = n$. 

\subsection{Countably Infinite Sets}

$X$ is \emph{countable infinite} if $\abs{X} = \abs{\ZZ_+}$. $X$ is \emph{countable} if it is either finite or countably infinite.

$X$ is \emph{uncountable} if it is not countable.

\subsection{Composition}

$\varphi : A \to B, \Psi : B \to C$. Then $\Psi \circ \varphi : A \to C$, $(\Psi \circ \varphi)(x) = \Psi(\varphi(x))$.

\subsection{Inverse Maps}

Only bijective maps have inverses.

\subsection{Binary Relations}

A \emph{binary relation} is a map $$* : A \times A \to A$$ It is called \emph{associative} if order of parentheses don't matter. It is called \emph{commutative} if order of elements doesn't matter. 

\subsection{Equivalence Relations}

An \emph{equivalence relation} is a relation that is reflexive ($a \sim a$), symmetric ($a \sim b \implies b \sim A$), and transitive (if $a \sim b$ and $b \sim c$, then $a \sim c$).

\subsection{Equivalence Classes}

Let $A$ be a set. Then $[a] = \{x \in A \, | \, x \sim a\}$ . $[a]$ is a representative of a class. A \emph{partition} of a set $X$ such that each subset is nonempty, disjoint, and all the subsets cover the entire set. It is then a collection of \emph{nonempty}, \emph{disjoint} subsets of $X$ that covers $X$.

The set of equivalence classes forms a partition.

\subsection{Ordered Set}

An ordered set is a set with an \emph{order}.

Notation: $\{a,b,c\}$ - unordered; $(a,b,c)$ - ordered

\subsection{Cartesian Product}
\subsection{Groups and Subgroups}
\subsection{Cyclic Groups}
\subsection{Permutation Groups} 

\section{Linear Algebra in One Lecture}

\subsection{Fields}

An \emph{associative ring} is a set $R$ woth two binary operations, addition and multiplication that are both \emph{associative} and satisfy 1) distributivity , 2) has an additive identity 3) every element has an additive inverse. 

If $R$ has multiplicative identity called 1, $R$ is called an \emph{associative ring with unit}. If every nonzero element has multiplicative inverse, then $R$ is a \emph{division ring}. If multiplication in a division ring is commutative, then $R$ is a called a \emph{field.}

\subsection{Linear Algebra} (see my notes for 454 and for Commutative Algebra as needed)

\begin{defn}\label{def1-0}
A \emph{vector space over a field $\FF$} is a set $V$ with an associative binary operation $$+ : V \times V \to V$$ called \emph{addition of vectors} and an operation $$\cdot : \FF \times V \to V$$ called \emph{scalar multiplication} ($(u,v) \mapsto u + v$, and$(a,v) \mapsto a\cdot v$) such that every element has 

\begin{enumerate}
\item additive inverse 
\item addition and multiplication are associative
\item multiplicative identity
\item distributivity of multiplication over addition. 
\end{enumerate}
\end{defn}

\begin{defn}\label{def1-1}
Let $B \subset V$ be a subset of $V$. Then $$\spann{B} = \{\sum_{i=1}^m{a_iv_i} \, | \, a_i \in \FF, v_i \in B\}$$ is the set of all linear combinations of vectors in $B$.
\end{defn}

\begin{prop}\label{prop0}
$\spann{B}$ is a vector subspace.

\begin{proof}

\end{proof}
\end{prop}

\begin{defn}\label{defn1-2}
A subset $B \subset V$ is called a \emph{spanning set} of $V$ if it \emph{spans $V$}, i.e. $\spann{B} = V$. 
\end{defn}

Other topics to fill out as needed from linear algebra:

\begin{enumerate}
\item linear independence
\item basis
\item dimension of a finite dimensional vector space
\item maps (homomorphisms)
\item kernels
\item kernel is a vector subspace
\item nullity
\item images
\item rank
\item quotient spaces
\end{enumerate}

\begin{thm}{(Rank-Nullity Theorem)}\label{thm0}
$$dim{(Ker ({T}))} + dim{(Im ({T}))} = dim{(V)}$$ and $$null {(T)} + rank (T) = dim{(V)}$$
\end{thm}


\end{document}